% Options for packages loaded elsewhere
\PassOptionsToPackage{unicode}{hyperref}
\PassOptionsToPackage{hyphens}{url}
%
\documentclass[
]{article}
\usepackage{amsmath,amssymb}
\usepackage{lmodern}
\usepackage{iftex}
\ifPDFTeX
  \usepackage[T1]{fontenc}
  \usepackage[utf8]{inputenc}
  \usepackage{textcomp} % provide euro and other symbols
\else % if luatex or xetex
  \usepackage{unicode-math}
  \defaultfontfeatures{Scale=MatchLowercase}
  \defaultfontfeatures[\rmfamily]{Ligatures=TeX,Scale=1}
\fi
% Use upquote if available, for straight quotes in verbatim environments
\IfFileExists{upquote.sty}{\usepackage{upquote}}{}
\IfFileExists{microtype.sty}{% use microtype if available
  \usepackage[]{microtype}
  \UseMicrotypeSet[protrusion]{basicmath} % disable protrusion for tt fonts
}{}
\makeatletter
\@ifundefined{KOMAClassName}{% if non-KOMA class
  \IfFileExists{parskip.sty}{%
    \usepackage{parskip}
  }{% else
    \setlength{\parindent}{0pt}
    \setlength{\parskip}{6pt plus 2pt minus 1pt}}
}{% if KOMA class
  \KOMAoptions{parskip=half}}
\makeatother
\usepackage{xcolor}
\usepackage[margin=1in]{geometry}
\usepackage{color}
\usepackage{fancyvrb}
\newcommand{\VerbBar}{|}
\newcommand{\VERB}{\Verb[commandchars=\\\{\}]}
\DefineVerbatimEnvironment{Highlighting}{Verbatim}{commandchars=\\\{\}}
% Add ',fontsize=\small' for more characters per line
\usepackage{framed}
\definecolor{shadecolor}{RGB}{248,248,248}
\newenvironment{Shaded}{\begin{snugshade}}{\end{snugshade}}
\newcommand{\AlertTok}[1]{\textcolor[rgb]{0.94,0.16,0.16}{#1}}
\newcommand{\AnnotationTok}[1]{\textcolor[rgb]{0.56,0.35,0.01}{\textbf{\textit{#1}}}}
\newcommand{\AttributeTok}[1]{\textcolor[rgb]{0.77,0.63,0.00}{#1}}
\newcommand{\BaseNTok}[1]{\textcolor[rgb]{0.00,0.00,0.81}{#1}}
\newcommand{\BuiltInTok}[1]{#1}
\newcommand{\CharTok}[1]{\textcolor[rgb]{0.31,0.60,0.02}{#1}}
\newcommand{\CommentTok}[1]{\textcolor[rgb]{0.56,0.35,0.01}{\textit{#1}}}
\newcommand{\CommentVarTok}[1]{\textcolor[rgb]{0.56,0.35,0.01}{\textbf{\textit{#1}}}}
\newcommand{\ConstantTok}[1]{\textcolor[rgb]{0.00,0.00,0.00}{#1}}
\newcommand{\ControlFlowTok}[1]{\textcolor[rgb]{0.13,0.29,0.53}{\textbf{#1}}}
\newcommand{\DataTypeTok}[1]{\textcolor[rgb]{0.13,0.29,0.53}{#1}}
\newcommand{\DecValTok}[1]{\textcolor[rgb]{0.00,0.00,0.81}{#1}}
\newcommand{\DocumentationTok}[1]{\textcolor[rgb]{0.56,0.35,0.01}{\textbf{\textit{#1}}}}
\newcommand{\ErrorTok}[1]{\textcolor[rgb]{0.64,0.00,0.00}{\textbf{#1}}}
\newcommand{\ExtensionTok}[1]{#1}
\newcommand{\FloatTok}[1]{\textcolor[rgb]{0.00,0.00,0.81}{#1}}
\newcommand{\FunctionTok}[1]{\textcolor[rgb]{0.00,0.00,0.00}{#1}}
\newcommand{\ImportTok}[1]{#1}
\newcommand{\InformationTok}[1]{\textcolor[rgb]{0.56,0.35,0.01}{\textbf{\textit{#1}}}}
\newcommand{\KeywordTok}[1]{\textcolor[rgb]{0.13,0.29,0.53}{\textbf{#1}}}
\newcommand{\NormalTok}[1]{#1}
\newcommand{\OperatorTok}[1]{\textcolor[rgb]{0.81,0.36,0.00}{\textbf{#1}}}
\newcommand{\OtherTok}[1]{\textcolor[rgb]{0.56,0.35,0.01}{#1}}
\newcommand{\PreprocessorTok}[1]{\textcolor[rgb]{0.56,0.35,0.01}{\textit{#1}}}
\newcommand{\RegionMarkerTok}[1]{#1}
\newcommand{\SpecialCharTok}[1]{\textcolor[rgb]{0.00,0.00,0.00}{#1}}
\newcommand{\SpecialStringTok}[1]{\textcolor[rgb]{0.31,0.60,0.02}{#1}}
\newcommand{\StringTok}[1]{\textcolor[rgb]{0.31,0.60,0.02}{#1}}
\newcommand{\VariableTok}[1]{\textcolor[rgb]{0.00,0.00,0.00}{#1}}
\newcommand{\VerbatimStringTok}[1]{\textcolor[rgb]{0.31,0.60,0.02}{#1}}
\newcommand{\WarningTok}[1]{\textcolor[rgb]{0.56,0.35,0.01}{\textbf{\textit{#1}}}}
\usepackage{graphicx}
\makeatletter
\def\maxwidth{\ifdim\Gin@nat@width>\linewidth\linewidth\else\Gin@nat@width\fi}
\def\maxheight{\ifdim\Gin@nat@height>\textheight\textheight\else\Gin@nat@height\fi}
\makeatother
% Scale images if necessary, so that they will not overflow the page
% margins by default, and it is still possible to overwrite the defaults
% using explicit options in \includegraphics[width, height, ...]{}
\setkeys{Gin}{width=\maxwidth,height=\maxheight,keepaspectratio}
% Set default figure placement to htbp
\makeatletter
\def\fps@figure{htbp}
\makeatother
\setlength{\emergencystretch}{3em} % prevent overfull lines
\providecommand{\tightlist}{%
  \setlength{\itemsep}{0pt}\setlength{\parskip}{0pt}}
\setcounter{secnumdepth}{-\maxdimen} % remove section numbering
\ifLuaTeX
  \usepackage{selnolig}  % disable illegal ligatures
\fi
\IfFileExists{bookmark.sty}{\usepackage{bookmark}}{\usepackage{hyperref}}
\IfFileExists{xurl.sty}{\usepackage{xurl}}{} % add URL line breaks if available
\urlstyle{same} % disable monospaced font for URLs
\hypersetup{
  pdftitle={Problem Set 2 - Demonstrate the convergence of a t distribution},
  hidelinks,
  pdfcreator={LaTeX via pandoc}}

\title{Problem Set 2 - Demonstrate the convergence of a t distribution}
\author{}
\date{\vspace{-2.5em}}

\begin{document}
\maketitle

\hypertarget{group-3-sai-myint-yoojin-oh-liz-ward-karl-wirth}{%
\section{Group 3 : Sai Myint, Yoojin Oh, Liz Ward, Karl
Wirth}\label{group-3-sai-myint-yoojin-oh-liz-ward-karl-wirth}}

\begin{Shaded}
\begin{Highlighting}[]
\FunctionTok{library}\NormalTok{(ggplot2)}
\FunctionTok{library}\NormalTok{(cowplot)}
\end{Highlighting}
\end{Shaded}

\hypertarget{question-1}{%
\subsection{Question 1:}\label{question-1}}

\hypertarget{generate-50000-observations-from-a-random-t-gen-t03-rt3-distribution-with-the-following-degrees-of-freedom-3-6-12-27-59-and-also-a-standard-normal-distribution.}{%
\subsubsection{Generate 50,000 observations from a random t (gen t03=
rt(3)) distribution with the following degrees of freedom {[}3, 6, 12,
27, 59{]} and also a standard normal
distribution.}\label{generate-50000-observations-from-a-random-t-gen-t03-rt3-distribution-with-the-following-degrees-of-freedom-3-6-12-27-59-and-also-a-standard-normal-distribution.}}

\hfill\break
Setting up the plots for the t distributions and standard normal

\begin{Shaded}
\begin{Highlighting}[]
\CommentTok{\# Creating t distributions for each degree of freedom}
\NormalTok{t\_dist3 }\OtherTok{\textless{}{-}} \FunctionTok{rt}\NormalTok{(}\DecValTok{50000}\NormalTok{,}\DecValTok{3}\NormalTok{)}
\NormalTok{t\_dist6 }\OtherTok{\textless{}{-}} \FunctionTok{rt}\NormalTok{(}\DecValTok{50000}\NormalTok{,}\DecValTok{6}\NormalTok{)}
\NormalTok{t\_dist12 }\OtherTok{\textless{}{-}} \FunctionTok{rt}\NormalTok{(}\DecValTok{50000}\NormalTok{,}\DecValTok{12}\NormalTok{)}
\NormalTok{t\_dist27 }\OtherTok{\textless{}{-}} \FunctionTok{rt}\NormalTok{(}\DecValTok{50000}\NormalTok{,}\DecValTok{27}\NormalTok{)}
\NormalTok{t\_dist59 }\OtherTok{\textless{}{-}} \FunctionTok{rt}\NormalTok{(}\DecValTok{50000}\NormalTok{,}\DecValTok{59}\NormalTok{)}

\CommentTok{\# Distribution for standard normal}
\NormalTok{normal }\OtherTok{\textless{}{-}} \FunctionTok{rnorm}\NormalTok{(}\DecValTok{50000}\NormalTok{)}
\end{Highlighting}
\end{Shaded}

Generating the Plots for t distribution {[}3,6,12,27,59{]}

\begin{Shaded}
\begin{Highlighting}[]
\FunctionTok{hist}\NormalTok{(t\_dist3)}
\end{Highlighting}
\end{Shaded}

\includegraphics{ProblemSet2_files/figure-latex/unnamed-chunk-3-1.pdf}

\begin{Shaded}
\begin{Highlighting}[]
\FunctionTok{hist}\NormalTok{(t\_dist6)}
\end{Highlighting}
\end{Shaded}

\includegraphics{ProblemSet2_files/figure-latex/unnamed-chunk-3-2.pdf}

\begin{Shaded}
\begin{Highlighting}[]
\FunctionTok{hist}\NormalTok{(t\_dist12)}
\end{Highlighting}
\end{Shaded}

\includegraphics{ProblemSet2_files/figure-latex/unnamed-chunk-3-3.pdf}

\begin{Shaded}
\begin{Highlighting}[]
\FunctionTok{hist}\NormalTok{(t\_dist27)}
\end{Highlighting}
\end{Shaded}

\includegraphics{ProblemSet2_files/figure-latex/unnamed-chunk-3-4.pdf}

\begin{Shaded}
\begin{Highlighting}[]
\FunctionTok{hist}\NormalTok{(t\_dist59)}
\end{Highlighting}
\end{Shaded}

\includegraphics{ProblemSet2_files/figure-latex/unnamed-chunk-3-5.pdf}

\begin{Shaded}
\begin{Highlighting}[]
\FunctionTok{hist}\NormalTok{(normal)}
\end{Highlighting}
\end{Shaded}

\includegraphics{ProblemSet2_files/figure-latex/unnamed-chunk-3-6.pdf}

We can see that when the degrees of freedom increases in the
t-distribution it approaches a standard normal. With the degree of
freedom at 59, the distribution looks very close to a standard normal
distribution.

\hypertarget{two-tailed-95-scores-of-the-t-distribution}{%
\subsection{Two Tailed 95\% Scores of the T
Distribution}\label{two-tailed-95-scores-of-the-t-distribution}}

Determine the two tailed 95\% scores of the t distribution for each
degree of freedom and compare them to a z of the significance (1.96)

\hypertarget{calculating-the-two-tail-95-score}{%
\paragraph{Calculating the Two Tail 95\%
score}\label{calculating-the-two-tail-95-score}}

\begin{Shaded}
\begin{Highlighting}[]
\CommentTok{\# Z Score for the significance of 1.96}
\NormalTok{zscore }\OtherTok{\textless{}{-}} \FunctionTok{qt}\NormalTok{(}\FloatTok{0.95}\NormalTok{,}\DecValTok{1}\NormalTok{)}

\CommentTok{\# Calculating T scores for each Degrees of Freedom}
\NormalTok{tscore3 }\OtherTok{\textless{}{-}} \FunctionTok{qt}\NormalTok{(}\FloatTok{0.95}\NormalTok{,}\DecValTok{3}\NormalTok{)}
\NormalTok{tscore6 }\OtherTok{\textless{}{-}} \FunctionTok{qt}\NormalTok{(}\FloatTok{0.95}\NormalTok{,}\DecValTok{6}\NormalTok{)}
\NormalTok{tscore12 }\OtherTok{\textless{}{-}} \FunctionTok{qt}\NormalTok{(}\FloatTok{0.95}\NormalTok{,}\DecValTok{12}\NormalTok{)}
\NormalTok{tscore27 }\OtherTok{\textless{}{-}} \FunctionTok{qt}\NormalTok{(}\FloatTok{0.95}\NormalTok{,}\DecValTok{27}\NormalTok{)}
\NormalTok{tscore59 }\OtherTok{\textless{}{-}} \FunctionTok{qt}\NormalTok{(}\FloatTok{0.95}\NormalTok{,}\DecValTok{59}\NormalTok{)}

\CommentTok{\# Combinding all the t score for each DF to t\_scores to print all in one}
\NormalTok{t\_scores }\OtherTok{\textless{}{-}} \FunctionTok{c}\NormalTok{(tscore3,tscore6,tscore12,tscore27,tscore59)}

\CommentTok{\# Printing Scores}
\FunctionTok{print}\NormalTok{ (t\_scores)}
\end{Highlighting}
\end{Shaded}

\begin{verbatim}
## [1] 2.353363 1.943180 1.782288 1.703288 1.671093
\end{verbatim}

\begin{Shaded}
\begin{Highlighting}[]
\FunctionTok{print}\NormalTok{ (zscore)}
\end{Highlighting}
\end{Shaded}

\begin{verbatim}
## [1] 6.313752
\end{verbatim}

The Z score is significantly higher than the t scores of every
distribution. We expected this because the degrees of freedom is like a
safety net, and is a more conservative test than the standard normal
distribution.

\end{document}
